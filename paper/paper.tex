\documentclass{article}

\usepackage{arxiv}

\usepackage[utf8]{inputenc} % allow utf-8 input
\usepackage[T1]{fontenc}    % use 8-bit T1 fonts
\usepackage{lmodern}        % https://github.com/rstudio/rticles/issues/343
\usepackage{hyperref}       % hyperlinks
\usepackage{url}            % simple URL typesetting
\usepackage{booktabs}       % professional-quality tables
\usepackage{amsfonts}       % blackboard math symbols
\usepackage{nicefrac}       % compact symbols for 1/2, etc.
\usepackage{microtype}      % microtypography
\usepackage{graphicx}

\title{Automatically drawing vegetation maps using digital time-lapse
cameras in alpine ecosystems}

\author{
    Ryotaro Okamoto
    \thanks{\url{https://github.com/0kam}}
   \\
    Doctoral Program in Biology \\
    University of Tsukuba \\
  1-1-1 Tennodai, Tsukuba, Ibaraki 305-8577 Japan. \\
  \texttt{\href{mailto:okamoto.ryotaro.su@alumni.tsukuba.ac.jp}{\nolinkurl{okamoto.ryotaro.su@alumni.tsukuba.ac.jp}}} \\
   \And
    Hiroyuki Oguma
   \\
    Biodiversity Division \\
    National Institute for Environmental Studies \\
  16-2 Onogawa, Tsukuba, Ibaraki 305-8506 JAPAN \\
  \texttt{\href{mailto:oguma@nies.go.jp}{\nolinkurl{oguma@nies.go.jp}}} \\
   \And
    Reiko Ide
   \\
    Earth System Division \\
    National Institute for Environmental Studies \\
  16-2 Onogawa, Tsukuba, Ibaraki 305-8506 JAPAN \\
  \texttt{\href{mailto:ide.reiko@nies.go.jp}{\nolinkurl{ide.reiko@nies.go.jp}}} \\
  }


% tightlist command for lists without linebreak
\providecommand{\tightlist}{%
  \setlength{\itemsep}{0pt}\setlength{\parskip}{0pt}}

% From pandoc table feature
\usepackage{longtable,booktabs,array}
\usepackage{calc} % for calculating minipage widths
% Correct order of tables after \paragraph or \subparagraph
\usepackage{etoolbox}
\makeatletter
\patchcmd\longtable{\par}{\if@noskipsec\mbox{}\fi\par}{}{}
\makeatother
% Allow footnotes in longtable head/foot
\IfFileExists{footnotehyper.sty}{\usepackage{footnotehyper}}{\usepackage{footnote}}
\makesavenoteenv{longtable}


\usepackage{amsmath}
\setlength{\parindent}{10.5pt}
\begin{document}
\maketitle


\begin{abstract}
Enter the text of your abstract here.
\end{abstract}

\keywords{
    alpine ecology
   \and
    deep learning
   \and
    ecosystem monitoring
  }

\hypertarget{introduction}{%
\section{Introduction}\label{introduction}}

The effects of climate change on terrestrial ecosystems are particularly
significant in alpine regions (IPCC 2007). Alpine vegetation depends on
severe climatic conditions such as low temperatures and long
snow-covered periods. Thus alpine areas have rare and unique species
adapted to the extreme environments. Several studies have reported that
recent global climatic changes, e.g., increasing temperatures and
reducing snow-covered periods, have accelerated the invasion of
non-native species into alpine areas (see Alexander et al., 2016). In
Japan, dwarf bamboo (\emph{Sasa kurilensis}) has invaded alpine snow
meadows, probably driven by the extension of the snow-free period (Kudo
et al., 2011). Also, climate change has affected the growth and
phenologies of native species. For example, the growth of dwarf pine
(\emph{Pinus pumila}), a dominant species in Japanese alpine regions,
has been affected by climatic conditions such as temperature and
snowmelt (Amagai et al., 2015). Monitoring and predicting such changes
are essential for effective conservation planning. Since the impact of
climate change on alpine vegetation varies depending on species and the
microhabitats (Kudo et al., 2010), spatially high-resolution monitoring
with a wide range is required.

Previous studies have mainly depended on field observations, yet it is
hard to cover broad areas in alpine regions due to poor accessibility
and severe weather. Satellite, airborne, and Unmanned Aerial Vehicle
(UAV) remote sensing methods seem to be alternatives. However, satellite
imageries of alpine areas are rarely available due to cloud cover, and
the spatial resolution is not enough to observe vegetation changes at
the plant community scale. Airborne imageries can obtain high-resolution
data, but its cost becomes a bottleneck for frequent monitoring.
Although UAV methods have become popular as a cost-effective tool for
ecological monitoring (citation), operating UAVs in alpine regions is
challenging due to the strong wind and harsh topology.

On the other hand, researchers have also utilized automated digital
time-lapse cameras mounted on the ground for monitoring green-leaf
phenologies in forests (Richardson et al., 2018), grasslands (Browning
et al., 2017) and alpine meadows (Ide and Oguma, 2013). Unlike satellite
imageries, such cameras provide images free of clouds and atmospheric
effects. Also, they can obtain high-resolution (i.e., sub-meter scale)
and frequent (i.e., daily or hourly) images at a meager cost. These
studies set some regions of interest (ROI) in the images and calculate
the phenology index (e.g., excess greenness, Woebbecke et al., 1995) for
each ROI. While we can visually interpret the vegetation distribution
and its changes from a set of such time-lapse images, few studies have
tackled this. This lack seems to be because applying such time-lapse
imageries in monitoring vegetation distribution has two technical
challenges.

First, unlike multispectral sensors satellites equip, digital cameras
can only obtain three bands (Red, Green, and Blue), making it harder to
classify the vegetation. Second, since digital time-lapse cameras are
mounted on the ground, applying geo-spatial analysis to these images is
challenging. In other words, even if we find a vegetation change, we
cannot identify its geographic location and analyze it with topological
data. Treating ground-based time-lapse images as geographic data is
essential for utilizing these in conservation planning.

This study proposes an automated method for drawing vegetation maps and
locating vegetation changes with a digital time-lapse camera by solving
these two challenges. We aim to use cheap but powerful digital
time-lapse cameras in alpine ecosystem conservation.

\hypertarget{materials-and-methods}{%
\section{Materials and methods}\label{materials-and-methods}}

\hypertarget{digital-time-lapse-camera-imagery}{%
\subsection{Digital time-lapse camera
imagery}\label{digital-time-lapse-camera-imagery}}

We used repeat photography data owned by National Institute for
Environmental Studies, Japan (NIES). All the images are publically
available on NIES' webpage
(\url{https://db.cger.nies.go.jp/gem/ja/mountain/station.html?id=2}). In
2010, NIES installed the digital time-lapse camera (EOS 5D MK2, Canon
Inc., 21 M pixels) on a mountain lodge Murodo-sanso (about 2350 m
a.s.l., above the forest limit), located at the foot of Mt. Tateyama
(3015 m a.s.l.), in the Nothern Japanese Alps. The camera takes one
photograph per hour, from 6 a.m. to 7 p.m. . The camera's field of view
(FOV) includes Mt. Tateyama, which ranges from about 2350 m a.s.l. to
3015 m a.s.l. in elevation. The area has a complex mosaic-like
vegetation structure because of its topography, including rocks, cliffs,
curls, and moraines. From April to November, the camera has observed the
snowmelt and seasonal phenology of evergreen, deciduous dwarf trees
(e.g., Pinus pumila, Sorbus sp), dwarf bamboos (e.g., Sasa kurilensis),
and alpine herbaceous plants (e.g., Geum pentapetalum, Nephrophyllidium
crista-galli). We pulled the images from late summer to late fall of
2010 and 2020.

\hypertarget{preprocessing}{%
\subsection{Preprocessing}\label{preprocessing}}

\hypertarget{selecting-images}{%
\subsubsection{Selecting images}\label{selecting-images}}

First, we selected images that are suitable for vegetation
classification. We choose four days with good weather from late summer
to late fall each year (9/19, 10/1, 10/11, 10/23 in 2010. 9/19, 10/2,
10/11, 10/23 in 2020). We used the images of this season because we can
separate vegetation from the patterns of autumn foliage coloration. Then
we pull four images from 11 a.m. to 2 p.m. each day to avoid the effect
of temporal noise such as shadows. Finally, we got 16 images for each
year.

\hypertarget{automatic-image-to-image-alignment}{%
\subsubsection{Automatic image-to-image
alignment}\label{automatic-image-to-image-alignment}}

Since images are slightly misaligned with each other due to wind, we
aligned them before processing. We implemented the program with Python3
language and OpenCV4 image processing library. First, we set one image
of 2010 as the alignment target. Next, we automatically found matching
keypoints between the target and other images using AKAZE local feature
extractor and K-nearest neighbor matcher. Then, we searched and applied
the homography matrix that minimizes the distance between each pair of
the matching points.

\hypertarget{automatic-vegetation-classification}{%
\subsection{Automatic vegetation
classification}\label{automatic-vegetation-classification}}

Next, we classified the pixel time series of repeat photography
imageries into vegetations. Because deciduous plants have great
differences in autumn phenology among species, researchers have used
that information for vegetation classification with satellite imageries
(e.g., Tigges et al., 2013, Son et al., 2014, Heupel et al., 2018). In
our target area, we can recognize the vegetation type with the pixel
time series of autumn (Fig. 2). However, no research has applied this
technique to ground-based repeat photography imageries. We developed a
deep-learning method to take advantage of high-resolution and frequent
repeat photography data. Fig.3 shows the classification process.

\hypertarget{model-architecture}{%
\subsubsection{Model Architecture}\label{model-architecture}}

Since ground-based photographs can provide high spatial resolution
(about \textasciitilde{} 0.5m in our dataset), we can use leaf texture
as additional information for classification. For example, we can see a
glossy surface on dwarf bamboos and a mat surface on dwarf pines. Thus,
we used a small patch (9x9 pixels square) rather than a single pixel as
an input of the model. Therefore, a model input is a time series of
image patches. Our model has two components to deal with this data
structure (see Fig. 3). First, we extracted each patch's features (e.g.,
texture) using Convolutional Neural Network (CNN) layers (Fig. 4a). CNN
is a neural network specialized in recognizing spatial structures of
data, such as images. The CNN part outputs a time series of extracted
features. Second, we used Recurrent Neural Net (RNN) layers (Fig. 4b) to
classify the temporal patterns of the features extracted by the CNN
part. RNN is a neural network specialized in recognizing temporal or
sequential data dynamics, such as text and speech. Amongst many variants
of RNN, we used Long Short Time Memory (LSTM). Combining CNN and RNN,
our model can classify the input image-patch time series considering the
colors and textures of each patch and their temporal patterns. This
CNN-RNN architecture has also been used for audio classification. Using
a Deep Learning method has another positive side effect; mini-batch
training. Since we used high-resolution images (16 x 21M pix), the
training dataset became so immense that we could not feed them to the
model at once: that consumes the computer's RAM too much. However, using
Deep Learning methods enables us to provide them in small portions
(called mini-batches) to save RAM consumption.

\hypertarget{implementation-and-model-training}{%
\subsubsection{Implementation and model
training}\label{implementation-and-model-training}}

We implemented the classifier with Python3 language and PyTorch deep
neural network library. All source codes are publically available via
GitHub (\url{https://github.com/0kam/xxxx}).

\hypertarget{dataset-preparation}{%
\subsubsection{Dataset preparation}\label{dataset-preparation}}

We set 5 classes: dwarf pine, dwarf bamboo, other vegetation, no
vegetation, and sky. Using free image annotation software (Semantic
Segmentation Editor), an expert prepared teacher data (Fig. x) for
classification.

\hypertarget{automatic-georectification}{%
\section{Automatic georectification}\label{automatic-georectification}}

Then, we developed a novel method to convert ground-based landscape
imagery into GIS-ready geographical data. This process is called
georectification. Georectification of ground-based images has been a
difficult task, and this causes the underuse of potentially rich
information in ground-based imageries. In plain words, georectification
means aligning images into Digital Surface Models (DSMs) so that every
pixel of an image gets a geographical coordinate. We can consider a
camera as a function that transforms 3D geographical coordinates
(latitude, longitude, height) into 2D image coordinates (locations of
each pixel). So estimating the parameters of this function (such as the
camera location, pose, and the field of view), we can align an image to
a DSM. Usually, georectification has three steps: 1. Finding Ground
Control Points (GCPs) in the image. 2. Estimating camera parameters such
as camera poses and field of view using GCPs. 3. Mapping the image onto
the DSM using camera parameters.

Recently, researchers have developed some georectification methods to
use ground-based photographs in glaciology (Messerli and Grinsted 2015)
and snow cover studies (Portnier et al.~2020). Especially, Portnier et
al.~2020 is worth mentioning for its semi-automatic method using
mountain silhouettes as GCPs. However, this silhouette-based method has
a drawback in the projection accuracy. It only uses limited areas
(silhouettes) of images in the image-to-DSM alignment, and also it
ignores lens distortion. Because our target site has a complex
vegetation distribution and our camera has considerable lens distortion,
we needed a more accurate method.

\hypertarget{local-feature-based-matching-of-images-and-dsms}{%
\subsection{Local-feature-based matching of images and
DSMs}\label{local-feature-based-matching-of-images-and-dsms}}

To get matching points between images and DSMs on a broader area, we
used an airborne image already georectified. Combining an airborne image
and a DSM, we rendered simulated landscape images (Fig. x). Then we got
matching points (GCPs) by applying AKAZE local feature matcher to a
target image and this simulated image (Fig. x). GCPs have geographical
coordinates (from the DSM) and image coordinates (from the image). Our
procedure requires the camera's exact location and initial camera
parameters to render the simulated image.

\hypertarget{modeling-and-estimating-lens-distortions}{%
\subsection{Modeling and estimating lens
distortions}\label{modeling-and-estimating-lens-distortions}}

We modeled the lens distortion based on Weng et al.~1992 and OpenCV's
implementation (Eq. 1). Our model includes radial
(k1\textasciitilde k6), tangental (p1, p2), thin prism
(s1\textasciitilde s4) distortion, and unequal pixel aspect ratio (a1,
a2). See Weng et al.~1992 for lens distortion modeling. Also, we
recommend readers to Portnier et al.~2020 for camera models without lens
distortions. Thus, now we can project GCPs' geographical coordinates
into image coordinates using lens distortion parameters and other camera
parameters (the camera location, pose, and the field of view). We
optimized these (except camera location) by minimizing the square
projection error of the GCPs using the Covariance Matrix Adaptation
Evolution Strategy (CMA-ES). We could not estimate the camera location
because it makes the problem too complicated (e.g., a telephoto taken
from a distance and a wide-angle taken from a close look similar).

\[
\begin{bmatrix}
{x}'' \\ 
{y}'' 
\end{bmatrix} 
= 
\begin{bmatrix} 
x’ \frac{1 + k_1 r^2 + k_2 r^4 + k_3 r^6}{1 + k_4 r^2 + k_5 r^4 + k_6 r^6} + 2 p_1 x’ y’ + p_2(r^2 + 2 x’^2) + s_1 r^2 + s_2 r^4 \\ 
y’ \frac{1 + a_1 + k_1 r^2 + k_2 r^4 + k_3 r^6}{1 + a_2 + k_4 r^2 + k_5 r^4 + k_6 r^6} + p_1 (r^2 + 2 y’^2) + 2 p_2 x’ y’ + s_3 r^2 + s_4 r^4 \ \end{bmatrix}
\]

\hypertarget{implementation-and-data-set}{%
\subsection{Implementation and data
set}\label{implementation-and-data-set}}

We implemented the algorithm with Python3 language and published it as
an open-source package via GitHub
(\url{https://github.com/0kam/alproj}). You can try it with your data.
We used an airborne photograph taken in the November of 201X with a
spatial resolution of 1.0 m. Also, we used the 5m resolution Digital
Elevation Model provided by the Geospatial Information Authority of
Japan (GSI) as a DSM.

\hypertarget{results}{%
\section{Results}\label{results}}

\hypertarget{vegetation-classification-accuracy}{%
\subsection{Vegetation classification
accuracy}\label{vegetation-classification-accuracy}}

\begin{longtable}[]{@{}
  >{\raggedright\arraybackslash}p{(\columnwidth - 14\tabcolsep) * \real{0.05}}
  >{\centering\arraybackslash}p{(\columnwidth - 14\tabcolsep) * \real{0.14}}
  >{\centering\arraybackslash}p{(\columnwidth - 14\tabcolsep) * \real{0.16}}
  >{\centering\arraybackslash}p{(\columnwidth - 14\tabcolsep) * \real{0.12}}
  >{\centering\arraybackslash}p{(\columnwidth - 14\tabcolsep) * \real{0.13}}
  >{\centering\arraybackslash}p{(\columnwidth - 14\tabcolsep) * \real{0.16}}
  >{\centering\arraybackslash}p{(\columnwidth - 14\tabcolsep) * \real{0.15}}
  >{\centering\arraybackslash}p{(\columnwidth - 14\tabcolsep) * \real{0.09}}@{}}
\toprule
\begin{minipage}[b]{\linewidth}\raggedright
\end{minipage} & \begin{minipage}[b]{\linewidth}\centering
\textbf{Macro Average}, N = 5
\end{minipage} & \begin{minipage}[b]{\linewidth}\centering
\textbf{Weighted Average}, N = 5
\end{minipage} & \begin{minipage}[b]{\linewidth}\centering
\textbf{Dwarf Pine}, N = 5
\end{minipage} & \begin{minipage}[b]{\linewidth}\centering
\textbf{Dwarf Bamboo}, N = 5
\end{minipage} & \begin{minipage}[b]{\linewidth}\centering
\textbf{Other Vegetations}, N = 5
\end{minipage} & \begin{minipage}[b]{\linewidth}\centering
\textbf{Non Vegetation}, N = 5
\end{minipage} & \begin{minipage}[b]{\linewidth}\centering
\textbf{Sky}, N = 5
\end{minipage} \\
\midrule
\endhead
Precision & 0.977 (0.008) & 0.997 (0.001) & 0.977 (0.012) & 0.951
(0.013) & 0.999 (0.000) & 0.959 (0.066) & 1.000 (0.000) \\
Recall & 0.987 (0.001) & 0.997 (0.001) & 0.995 (0.003) & 0.960 (0.006) &
0.996 (0.002) & 0.984 (0.004) & 1.000 (0.000) \\
F1 Score & 0.982 (0.005) & 0.997 (0.001) & 0.986 (0.005) & 0.955 (0.004)
& 0.997 (0.001) & 0.970 (0.034) & 1.000 (0.000) \\
\bottomrule
\end{longtable}

\hypertarget{georectification-accuracy}{%
\subsection{Georectification accuracy}\label{georectification-accuracy}}

\hypertarget{vegetation-maps}{%
\subsection{Vegetation maps}\label{vegetation-maps}}

\hypertarget{detection-and-localization-of-vegetation-changes}{%
\subsection{Detection and localization of vegetation
changes}\label{detection-and-localization-of-vegetation-changes}}

\hypertarget{discussion}{%
\section{Discussion}\label{discussion}}

We suggested a fully automated procedure to transform time-lapse imagery
into georeferenced vegetation maps at a meager cost. This task is
challenging because of 1. the difficulty of classifying vegetations with
ordinal digital camera imagery and 2. the difficulty of
georectification. We solved these issues by 1. using the temporal
information of autumn leaf colors for vegetation classification and 2.
developing a novel method for accurate image georectification. Our
vegetation classification performance (with an accuracy of 0.87) and
georectification methods (with an RMSE of \(rmse\) m) are practical.

\hypertarget{the-benefit-of-using-time-series-imagery-for-vegetation-classification}{%
\subsection{The benefit of using time-series imagery for vegetation
classification}\label{the-benefit-of-using-time-series-imagery-for-vegetation-classification}}

One of the shortcomings of ordinal digital cameras is that they only
have three bands (Red, Blue, and Green). We made up for this lack of
information by using the rich temporal information that time-lapse
cameras can obtain. Many plant species have characteristic phenologies,
such as flowering and autumn foliages. Observing this requires long-term
(such as year-round) monitoring with a high frequency, and digital
time-lapse cameras are suitable. This study only focused on the two
species (dwarf bamboo and stone pine) that are easy to classify, even
with a single image. However, like the previous studies (Tigges et al.,
2013, Son et al., 2014, Heupel et al., 2018), we expect our method to
classify more vegetations (e.g., dwarf deciduous trees and alpine
herbaceous plants) using the temporal information of leaf colors. We
also used hourly images for each day. Fig. x shows that this additional
information made our model robust against transient noises, such as
shadows. Since alpine ecosystems have rough topologies, shadows are a
considerable problem. Time-lapse cameras are also beneficial for this
reason.

\hypertarget{the-performance-of-our-georectification-method-and-its-limitation.}{%
\subsection{The performance of our georectification method and its
limitation.}\label{the-performance-of-our-georectification-method-and-its-limitation.}}

\hypertarget{future-application-and-conclusion}{%
\subsection{Future application and
conclusion}\label{future-application-and-conclusion}}

\bibliographystyle{unsrt}
\bibliography{references.bib}


\end{document}
