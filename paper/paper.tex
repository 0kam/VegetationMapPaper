\documentclass{article}

\usepackage{arxiv}

\usepackage[utf8]{inputenc} % allow utf-8 input
\usepackage[T1]{fontenc}    % use 8-bit T1 fonts
\usepackage{lmodern}        % https://github.com/rstudio/rticles/issues/343
\usepackage{hyperref}       % hyperlinks
\usepackage{url}            % simple URL typesetting
\usepackage{booktabs}       % professional-quality tables
\usepackage{amsfonts}       % blackboard math symbols
\usepackage{nicefrac}       % compact symbols for 1/2, etc.
\usepackage{microtype}      % microtypography
\usepackage{graphicx}

\title{Automatically drawing vegetation maps using digital time-lapse cameras in alpine ecosystems}

\author{
    Ryotaro Okamoto
    \thanks{\url{https://github.com/0kam}}
   \\
    Doctoral Program in Biology \\
    University of Tsukuba \\
  1-1-1 Tennodai, Tsukuba, Ibaraki 305-8577 Japan. \\
  \texttt{\href{mailto:okamoto.ryotaro.su@alumni.tsukuba.ac.jp}{\nolinkurl{okamoto.ryotaro.su@alumni.tsukuba.ac.jp}}} \\
   \And
    Hiroyuki Oguma
   \\
    Biodiversity Division \\
    National Institute for Environmental Studies \\
  16-2 Onogawa, Tsukuba, Ibaraki 305-8506 JAPAN \\
  \texttt{\href{mailto:oguma@nies.go.jp}{\nolinkurl{oguma@nies.go.jp}}} \\
   \And
    Reiko Ide
   \\
    Earth System Division \\
    National Institute for Environmental Studies \\
  16-2 Onogawa, Tsukuba, Ibaraki 305-8506 JAPAN \\
  \texttt{\href{mailto:ide.reiko@nies.go.jp}{\nolinkurl{ide.reiko@nies.go.jp}}} \\
  }


% tightlist command for lists without linebreak
\providecommand{\tightlist}{%
  \setlength{\itemsep}{0pt}\setlength{\parskip}{0pt}}

% From pandoc table feature
\usepackage{longtable,booktabs,array}
\usepackage{calc} % for calculating minipage widths
% Correct order of tables after \paragraph or \subparagraph
\usepackage{etoolbox}
\makeatletter
\patchcmd\longtable{\par}{\if@noskipsec\mbox{}\fi\par}{}{}
\makeatother
% Allow footnotes in longtable head/foot
\IfFileExists{footnotehyper.sty}{\usepackage{footnotehyper}}{\usepackage{footnote}}
\makesavenoteenv{longtable}


\usepackage{amsmath}
\setlength{\parindent}{10.5pt}
\usepackage{booktabs}
\usepackage{longtable}
\usepackage{array}
\usepackage{multirow}
\usepackage{wrapfig}
\usepackage{float}
\usepackage{colortbl}
\usepackage{pdflscape}
\usepackage{tabu}
\usepackage{threeparttable}
\usepackage{threeparttablex}
\usepackage[normalem]{ulem}
\usepackage{makecell}
\usepackage{xcolor}
\begin{document}
\maketitle


\begin{abstract}
Enter the text of your abstract here.
\end{abstract}

\keywords{
    alpine ecology
   \and
    deep learning
   \and
    ecosystem monitoring
  }

\hypertarget{introduction}{%
\section{Introduction}\label{introduction}}

The effects of climate change on terrestrial ecosystems are particularly significant in alpine regions (\cite{IPCC2007}). Alpine vegetation depends on severe conditions such as low temperatures and long snow-covered periods. Thus alpine areas have rare and unique species adapted to the extreme environments. Several studies have reported that recent global climatic changes, e.g., increasing temperatures and reducing snow-covered periods, have accelerated the invasion of non-native species into alpine areas (see \cite{Alexander2016AlpBotany}). In Japan, dwarf bamboo (\emph{Sasa kurilensis}) has invaded alpine snow meadows, probably driven by the extension of the snow-free period (\cite{Kudo2011EcoEvo}). Also, climate change has affected the growth and phenologies of native species. For example, the growth of dwarf pine (\emph{Pinus pumila}), a dominant species in Japanese alpine regions, has been affected by climatic conditions such as temperature and snowmelt (\cite{Amagai2015EcoRes}). Effective conservation planning requires early detection and constant monitoring of such vegetation changes. Also, since the impact of climate change on alpine vegetation varies depending on species and the microhabitats (\cite{Kudo2010AAA}), spatially high-resolution monitoring with a wide range is required.

Previous studies have mainly depended on field observations, yet it is hard to cover broad areas in alpine regions due to the poor accessibility and severe weather. Satellite, airborne, and Unmanned Aerial Vehicle (UAV) remote sensing methods seem to be alternatives. However, satellite imagery of alpine areas is rarely available due to cloud cover, and the spatial resolution is not enough to observe vegetation changes at the plant community scale. Airborne imagery can obtain high-resolution data, but its cost becomes a bottleneck for frequent monitoring. Although UAV methods have become a cost-effective tool for ecological monitoring (\cite{Baena2017PLOSONE}), operating UAVs in alpine regions is challenging due to the strong wind and harsh topology.

On the other hand, researchers have also utilized automated digital time-lapse cameras mounted on the ground for monitoring green-leaf phenologies in forests (\cite{Richardson2009EcolAppl}), grasslands (\cite{Browning2017RemSen}), and alpine meadows (\cite{IdeOguma2013EcolInfom}). Unlike satellite imagery, such cameras provide images free of clouds and atmospheric effects. Also, they can obtain high-resolution (i.e., sub-meter scale) and frequent (i.e., daily or hourly) images at a meager cost. These studies set some regions of interest (ROI) in the images and calculate the phenology index (e.g., excess greenness, \cite{Woebbecke1995ASAE}). However, few studies have utilized such images in monitoring vegetation distributions. This lack of studies seems to be because applying such repeat photography in monitoring vegetation distribution has two technical challenges.

First, unlike satellites' multispectral sensors, ordinal digital cameras can only obtain three bands (Red, Green, and Blue), making it harder to classify the vegetation. Second, since digital time-lapse cameras are mounted on the ground, transforming these images into geospatial data (e.g., orthoimage) is challenging. In other words, even if we classify the vegetation from such images, we cannot quantitatively measure and analyze it as geographic data. It is essential to treat ground-based images as geographic data to utilize them in conservation planning.

This study proposes an automated method for drawing vegetation maps with a digital time-lapse camera by solving these two challenges. We solved the first challenge by using time-lapse images to classify vegetation. Since the autumn leaf phenology varies among species, we utilized that information. We show the effectiveness of such phenological information in vegetation classification. We also developed a novel method for transforming a ground-based photograph into geographic data to tackle the second challenge. Finally, we show an example of a vegetation map drawn by the proposed method. We aim to use cheap but powerful digital time-lapse cameras in alpine ecosystem monitoring and conservation.

\hypertarget{materials-and-methods}{%
\section{Materials and methods}\label{materials-and-methods}}

\hypertarget{digital-time-lapse-camera-imagery}{%
\subsection{Digital time-lapse camera imagery}\label{digital-time-lapse-camera-imagery}}

We used repeat photography data owned by National Institute for Environmental Studies, Japan (NIES). All the images are publically available on NIES' webpage (\url{https://db.cger.nies.go.jp/gem/ja/mountain/station.html?id=2}). In 2010, NIES installed the digital time-lapse camera (EOS 5D MK2, Canon Inc., 21 M pixels) on a mountain lodge Murodo-sanso (about 2350 m a.s.l., above the forest limit), located at the foot of Mt. Tateyama (3015 m a.s.l.), in the Nothern Japanese Alps. The camera takes a photograph per hour, from 6 a.m. to 7 p.m. . The camera's field of view (FOV) includes Mt. Tateyama, which ranges from about 2350 m a.s.l. to 3015 m a.s.l. in elevation. The area has a complex mosaic-like vegetation structure because of its topography, including rocks, cliffs, curls, and moraines. From April to November, the camera has observed the snowmelt and seasonal phenology of evergreen(e.g., \emph{Pinus pumila}) and deciduous (\emph{Sorbus} sp., \emph{Betula ermanii}) dwarf trees, dwarf bamboos (e.g., \emph{Sasa kurilensis}), and alpine shrubs and herbaceous plants (e.g., \emph{Geum pentapetalum}, \emph{Nephrophyllidium crista-galli}). This study used images from the summer and fall of 2015 to classify vegetation with the temporal patterns of the leaf color.

\hypertarget{preprocessing}{%
\subsection{Preprocessing}\label{preprocessing}}

\hypertarget{selecting-images}{%
\subsubsection{Selecting images}\label{selecting-images}}

First, we selected images that are suitable for vegetation classification. We choose seven days with good weather from late summer to late fall of 2015 (8/25, 9/5, 9/12, 9/20, 9/26, 10/3, 10/10). We used the images of this season because we can separate vegetation from the patterns of autumn foliage coloration.

\hypertarget{automatic-image-to-image-alignment}{%
\subsubsection{Automatic image-to-image alignment}\label{automatic-image-to-image-alignment}}

Since images are slightly misaligned, we aligned them before processing. We implemented the program with the Python3 programming language and OpenCV4 (\url{https://opencv.org/}) image processing library. First, we set one image of 2015 as the alignment target. Next, we automatically found matching keypoints between the target and other images using the AKAZE local feature extractor (\cite{Alcantarilla2013AKAZE}) and K-nearest neighbor matcher. Then, we searched and applied the homography matrix that minimizes the distance between each pair of the matching points, using OpenCV's ``findHomography'' function. Applying the estimated homography matrix, we could align the images accurately (0.654 pixels in root mean square error of the matching points). Finally, we prepared an input mask to define image regions that should be ignored in the following procedure, such as the sky and the regions too close to the camera.

\hypertarget{automatic-vegetation-classification}{%
\subsection{Automatic vegetation classification}\label{automatic-vegetation-classification}}

Since we successfully aligned the images, we can stack the images and extract a time series of pixel values (Red, Green, and Blue, Fig. 1) for each pixel. Such pixel time series reflects the temporal patterns of leaf colors. Because deciduous plants have great differences in autumn phenology among species, researchers have used that information for vegetation classification with satellite imagery (e.g., \cite{Tigges2013RemSenEnv}, \cite{Son2013RemSen}, \cite{Heupel2018PFG}). However, no research has applied this technique to ground-based repeat photography imagery. We implemented Support Vector Machine (SVM) and Recurrent Neural Network (RNN) based vegetation classification methods in this study. We tested the effects of using pixel time series on the classification performance.

\hypertarget{model-architecture}{%
\subsubsection{Model Architecture}\label{model-architecture}}

We prepared two supervised models, SVM and RNN, to classify the pixel time series into vegetation categories. SVM is one of the most popular machine learning models in remote sensing and has many applications (\cite{Mountrakis2011SVMReview}), including vegetation classification with multitemporal satellite imagery (\cite{Tigges2013RemSenEnv}). RNN is a neural network that recognizes temporal or sequential data dynamics. Researchers have also utilized RNN for remote sensing tasks, such as land cover classification, with multi-temporal satellite imagery (\cite{Ienco2017RemSenLSTM}, \cite{Sharma2018NN}). Amongst many variants of RNN, we used Long Short Time Memory (LSTM, \cite{Hochreiter1997LSTM}), one of the most well-known RNN architectures. We also classified the pixel of every single image separately using SVM classifiers to test whether using multi-temporal imagery increases the classification performance.

\hypertarget{dataset-preparation}{%
\subsubsection{Dataset preparation}\label{dataset-preparation}}

We set 5 vegetation classes: dwarf pine, dwarf bamboo, other vegetation, no vegetation, and sky. Using a free image annotation software (Semantic Segmentation Editor, Hitachi, \url{https://github.com/Hitachi-Automotive-And-Industry-Lab/semantic-segmentation-editor}), an expert drew polygons on an image of 2015 for each class. Since vegetation might be different in 2012, 2015, and 2020, applying this teacher data to the image of 2012 and 2020 may cause misclassification. However, vegetation changes are expected to occur at the edge of the plant community. Thus we shrink each polygon with several pixels before applying it to the training.

\hypertarget{implementation-and-model-training}{%
\subsubsection{Implementation and model training}\label{implementation-and-model-training}}

We implemented the classifier with Python3 language and PyTorch deep neural network library (\url{https://pytorch.org/}). All source codes are publically available via GitHub (\url{https://github.com/0kam/xxxx}).

\hypertarget{automatic-georectification}{%
\section{Automatic georectification}\label{automatic-georectification}}

Then, we developed a novel method to convert ground-based landscape imagery into GIS-ready geographical data. This process is called georectification. Georectification of ground-based images has been a difficult task, and this causes the underuse of potentially rich information in ground-based imagery. In plain words, georectification means aligning images into Digital Surface Models (DSMs) so that every pixel of an image gets a geographical coordinate. We can consider a camera as a function that transforms 3D geographical coordinates (e.g., X, Y and height in a Universal Transverse Mercator coordinate system (UTM)) into 2D image coordinates (locations of each pixel in the picture). Estimating the parameters of this function (such as the camera location, pose, and the field of view), we can map an image onto a DSM. We recommend readers to \cite{Portenier2020Cryosphere} for a plain mathmatical explanation of this process. Usually, georectification has three steps:

\begin{enumerate}
\def\labelenumi{\arabic{enumi}.}
\tightlist
\item
  Finding Ground Control Points (GCPs) in the image.\\
\item
  Estimating camera parameters such as camera poses and field of view using GCPs.\\
\item
  Mapping the image onto the DSM using camera parameters.
\end{enumerate}

Recently, researchers have developed some georectification methods to use ground-based photographs in glaciology (\cite{Messerli2015GeoInst}) and snow cover studies (\cite{Portenier2020Cryosphere}). Especially, \cite{Portenier2020Cryosphere} is worth mentioning for its semi-automatic method using mountain silhouettes as GCPs. However, this silhouette-based method has a drawback in the projection accuracy. It only uses limited areas (silhouettes) of images in the image-to-DSM alignment, and also it ignores lens distortion. Because our target site has a complex vegetation distribution and our camera has considerable lens distortion, we needed a more accurate method.

\hypertarget{modeling-and-estimating-camera-parameters}{%
\subsection{Modeling and estimating camera parameters}\label{modeling-and-estimating-camera-parameters}}

As we mentioned before, we considered a camera as a function that transforms geographical coordinates of each location of the target mountain to image coordinates in the acquired photograph. We implemented this procedure using the OpenGL framework to accelerate it with graphical processing units (GPUs). In OpenGL, we can separate this process into three operations:

\begin{enumerate}
\def\labelenumi{\arabic{enumi}.}
\tightlist
\item
  Transforming the world (geographical) coordinates to the view coordinates using camera's extrinsic parameters (i.e., the location and angles of the camera).
\item
  Distorting the view coordinates using the lens distortion parameters.
\item
  Transforming the view coordinates to the screen (image) coordinates using camera's intrinsic parameters (i.e., the FoV and aspect ratio of the camera).
\end{enumerate}

First, we transform the absolute geographic coordinates (also known as the world coordinates) to the view coordinates that are relative to the camera's position and direction. We apply a \(4 \times 4\) view matrix (\ref{view_matrix}) to the geographic coordinates in this step. The view matrix represents the position and the direction (pan, tilt, roll) of the camera. Note that the geographic coordinate system must be cartesian (such as the UTM).

\label{view_matrix}
\begin{equation}
  M_{view} = 
  \begin{bmatrix}
    \cos roll & -\sin roll & 0 & 0 \\
    \sin roll & \cos roll & 0 & 0 \\
    0 & 0 & 1 & 0 \\
    0 & 0 & 0 & 1 \\
  \end{bmatrix}
  \begin{bmatrix}
    1 & 0 & 0 & 0 \\
    0 & \cos tilt & -\sin tilt & 0 \\
    0 & \sin tilt & \cos tilt & 0 \\
    0 & 0 & 0 & 1 \\
  \end{bmatrix}
  \begin{bmatrix}
    \cos pan & 0 & \sin pan & 0 \\
    0 & 1 & 0 & 0 \\
    -\sin pan & 0 & \cos pan & 0 \\
    0 & 0 & 0 & 1 \\
  \end{bmatrix}
  \begin{bmatrix}
    1 & 0 & 0 & -x \\
    0 & 1 & 0 & -z \\
    0 & 0 & 1 & -y \\
    0& 0 & 0 & 1 \\
  \end{bmatrix}
\end{equation}

Where \(pan, tilt, roll\) are the Euler angles of the camera and \(x, y, z\) are the camera location in the geographic coordinate system. Then, we can transform the geographic coordinates of the DSM \(\begin{bmatrix} X_{geo} & Z_{geo} & Y_{geo} & 1 \end{bmatrix}\) to the camera coordinates \(\begin{bmatrix} X_{camera} & Z_{camera} & Y_{camera} & 1 \end{bmatrix}\) applying the view matrix \(M_{view}\) (\ref{view_tf}). In the OpenGL's camera coordinate system, \(X_{camera}\), \(Z_{camera}\) and \(Y_{camera}\) represents horizontal, vertical, and depth positions respectively.

\label{view_tf}
\begin{equation}
  \begin{bmatrix} 
    X_{camera} \\ Z_{camera} \\ Y_{camera} \\ 1 
  \end{bmatrix}
  =
  M_{view}
  \begin{bmatrix} 
    X_{geo} \\ Z_{geo} \\ Y_{geo} \\ 1 
  \end{bmatrix}
\end{equation}

Second, we distort the camera coordinates to simulate the lens distortion. We modeled the lens distortion (\ref{dist_model}) based on \cite{Weng1992CameraCalib} and OpenCV's implementation, where \(X_{norm}\) and \(Z_{norm}\) are the \(Y\)-normalized coordinates of the camera coordinates. This model distorts points' locations according to the distance from the center of the image when projected to the image surface. Our model includes radial (\(k1~k6\)), tangental (\(p1, p2\)), thin prism (\(s1~s4\)) distortion, and unequal pixel aspect ratio (\(a1, a2\)). See Weng et al.~1992 for the details of lens distortion modeling.

\label{dist_model}
\begin{gather}
  X_{norm} = \frac{X_{camera}}{Y_{camera}} \\
  Z_{norm} = \frac{Z_{camera}}{Y_{camera}} \\
  r^2 = {X_{norm}}^2 + {Z_{norm}}^2 \\
  \begin{bmatrix}
    X_{dist\_norm} \\ 
    Z_{dist\_norm} \\
  \end{bmatrix} 
  = 
  \begin{bmatrix} 
    X_{norm} \frac{1 + k_1 r^2 + k_2 r^4 + k_3 r^6}{1 + k_4 r^2 + k_5 r^4 + k_6 r^6} + 2 p_1 x’ y’ + p_2(r^2 + 2 x’^2) + s_1 r^2 + s_2 r^4 \\ 
    Z_{norm} \frac{1 + a_1 + k_1 r^2 + k_2 r^4 + k_3 r^6}{1 + a_2 + k_4 r^2 + k_5 r^4 + k_6 r^6} + p_1 (r^2 + 2 y’^2) + 2 p_2 x’ y’ + s_3 r^2 + s_4 r^4 \\    \end{bmatrix} \\
  X_{dist} = X_{dist\_norm} Y_{camera} \\
  Z_{dist} = Z_{dist\_norm} Y_{camera} \\
\end{gather}

Then we transform the distorted camera coordinates to the image coordinates by perspective projection. During the perspective projection, a closer object is drawn larger. We used a \(4 \times 4\) projection matrix \(M_{proj}\) (\ref{proj_mat}), that represents the camera's horizontal (\(fov_x\)) and vertical (\(fov_z\)) FoV. Finally, we can get the image coordinates of each point of the DSM as \(X_{image}, Z_{image}\) (\ref{proj_tf}).

\label{proj_mat}
\begin{gather}
  f_x = \frac{1}{\frac{\tan fov_x}{2}} 
  f_z = \frac{1}{\frac{\tan fov_z}{2}}
  M_{proj} = 
  \begin{bmatrix} 
    f_x & 0 & 1 & 0\\ 
    0 & f_z & -1 & 0 \\ 
    0 & 0 & 0 & 1\\ 
    0 & 0 & -1 & 0 
  \end{bmatrix}
\end{gather}

\label{proj_tf}
\begin{equation}
  \begin{bmatrix} 
    X_{image} \\ Z_{image} \\ Y_{image} \\ 1 
  \end{bmatrix}
  =
  M_{proj}
  \begin{bmatrix} 
    X_{dist} \\ Z_{dist} \\ Y_{dist} \\ 1 
  \end{bmatrix}
\end{equation}

Thus, now we can project GCPs' geographical coordinates into image coordinates using lens distortion parameters and other camera parameters (the camera location, pose, and the field of view). We optimized these (except camera location) by minimizing the square projection error of the GCPs using the Covariance Matrix Adaptation Evolution Strategy (CMA-ES, \cite{Hansen2003CMAES}). We could not estimate the camera location because it makes the problem too complicated (e.g., a telephoto taken from a distance and a wide-angle taken from a close look similar).

\hypertarget{local-feature-based-matching-of-images-and-dsms}{%
\subsection{Local-feature-based matching of images and DSMs}\label{local-feature-based-matching-of-images-and-dsms}}

To get matching points between images and DSMs on a broader area, we used an airborne image already georectified. Combining an airborne image and a DSM, we rendered simulated landscape images (Fig. x). Then we got matching points (GCPs) by applying AKAZE local feature matcher to a target image and this simulated image (Fig. x). GCPs have geographical coordinates (from the DSM) and image coordinates (from the image). Our procedure requires the camera's exact location and initial camera parameters to render the simulated image.

\hypertarget{implementation-and-data-set}{%
\subsection{Implementation and data set}\label{implementation-and-data-set}}

We implemented the algorithm with Python3 language and published it as an open-source package via GitHub (\url{https://github.com/0kam/alproj}). You can try it with your data. We used an airborne photograph taken in the November of 201X with a spatial resolution of 1.0 m. Also, we used the 5m resolution Digital Elevation Model provided by the Geospatial Information Authority of Japan (GSI) as a DSM.

\hypertarget{results}{%
\section{Results}\label{results}}

\hypertarget{vegetation-classification-accuracy}{%
\subsection{Vegetation classification accuracy}\label{vegetation-classification-accuracy}}

We evaluated the performance of the vegetation classifier using a 5-fold cross-validation design. Each fold was stratified with the vegetation categories. We used three standard metrics in machine learning evaluation: precision, recall, and F1-score (\ref{metrics}), where \(TP, FP, FN\) is true positives, false positives, and false negatives, respectively.

\label{metrics}
\begin{align}
  precision = \frac{TP}{TP + FP} \\
  recall = \frac{TP}{TP + FN} \\
  F_1 = \frac{2 \cdot precision \cdot recall}{precision + recall} \\
\end{align}

The vegetation classifier achieved a high recall and precision in all five categories \ref{tab:vegetation_cv}. Recall and precision were lowest in dwarf bamboo with the smallest training data. Even in dwarf bamboo, false positives occurred at low frequency (about 5\%), and only a tiny amount of true positives were missed by the pipeline (about 4\%). Since the weighted average of each metric was higher (more than 0.99), balancing the training dataset with vegetation categories may improve these results.

\ref{fig:vegetation} shows the products of vegetation classification. We can observe the distribution of dwarf pine and dwarf bamboo at the plant-community scale.

```\#\{r, echo=FALSE, message=FALSE, fig.cap=``Cross Validation Results''\}
df \textless- read\_csv(``../runs/cnn\_lstm9x9\_cv\_5\_ep\_200/staritfied\_cv.csv'',
col\_select = c(metrics, vegetation, value, fold))
theme\_gtsummary\_mean\_sd()

df \%\textgreater\%
filter(metrics != ``support'') \%\textgreater\%
mutate(vegetation = translate\_classes(vegetation)) \%\textgreater\%
mutate(vegetation = factor(vegetation, levels = vegetation\_levels)) \%\textgreater\%
group\_by(., vegetation, metrics) \%\textgreater\%
summarise(
mean = sprintf(``\%1.3f'', mean(value)),
sd = sprintf(``(\%1.3f)'', sd(value))
) \%\textgreater\%
pivot\_longer(cols=c(mean, sd), names\_to=``var'') \%\textgreater\%
pivot\_wider(names\_from=metrics, values\_from=value) \%T\textgreater\%
write\_csv(``table1.csv'') \%\textgreater\%
select(-var) \%\textgreater\%
kable(format=``latex'', booktabs=TRUE, caption = ``Cross validation result \textbackslash label\{tab:vegetation\_cv\}'') \%\textgreater\%
kable\_styling() \%\textgreater\%
collapse\_rows(columns=1) \%\textgreater\%
footnote(general = ``Mean (SD) of 5-Fold Cross-Varidation'')

\begin{verbatim}
```#{r, vegetation, echo=FALSE, message=FALSE, fig.cap="Vegetation classification results", out.width="100%", auto_pdf = TRUE}
knitr::include_graphics("vegetation.jpg")
\end{verbatim}

\hypertarget{georectification-accuracy}{%
\subsection{Georectification accuracy}\label{georectification-accuracy}}

\hypertarget{vegetation-maps}{%
\subsection{Vegetation maps}\label{vegetation-maps}}

\hypertarget{detection-and-localization-of-vegetation-changes}{%
\subsection{Detection and localization of vegetation changes}\label{detection-and-localization-of-vegetation-changes}}

\hypertarget{discussion}{%
\section{Discussion}\label{discussion}}

We suggested a fully automated procedure to transform time-lapse imagery into georeferenced vegetation maps at a meager cost. This task is challenging because of 1. the difficulty of classifying vegetation with ordinal digital camera imagery and 2. the difficulty of georectification. We solved these issues by 1. using the temporal information of autumn leaf colors for vegetation classification and 2. developing a novel method for accurate image georectification. Our vegetation classification performance (with an accuracy of 0.87) and georectification methods (with an RMSE of \(rmse\) m) are practical.

\hypertarget{the-benefit-of-using-time-series-imagery-for-vegetation-classification}{%
\subsection{The benefit of using time-series imagery for vegetation classification}\label{the-benefit-of-using-time-series-imagery-for-vegetation-classification}}

One of the shortcomings of ordinal digital cameras is that they only have three bands (Red, Blue, and Green). We made up for this lack of information by using the rich temporal information that time-lapse cameras can obtain. Many plant species have characteristic phenologies, such as flowering and autumn foliages. Observing this requires long-term (such as year-round) monitoring with a high frequency, and digital time-lapse cameras are suitable. This study only focused on the two species (dwarf bamboo and stone pine) that are easy to classify, even with a single image. However, like the previous studies (\cite{Tigges2013RemSenEnv}, \cite{Son2013RemSen}, \cite{Heupel2018PFG}), we expect our method to classify more vegetation (e.g., dwarf deciduous trees and alpine herbaceous plants) using the temporal information of leaf colors. We also used hourly images for each day. Fig. x shows that this additional information made our model robust against transient noises, such as shadows. Since alpine ecosystems have rough topologies, shadows are a considerable problem. Time-lapse cameras are also beneficial for this reason.

\hypertarget{the-performance-of-the-georectification-method-and-its-limitation.}{%
\subsection{The performance of the georectification method and its limitation.}\label{the-performance-of-the-georectification-method-and-its-limitation.}}

\hypertarget{future-application-and-conclusion}{%
\subsection{Future application and conclusion}\label{future-application-and-conclusion}}

\bibliographystyle{unsrt}
\bibliography{export.bib}


\end{document}
